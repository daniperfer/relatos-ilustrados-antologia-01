Ethom James, un hombrecillo nervioso, calvo y con gruesas gafas de pasta, me habla rápido haciendo gestos con las manos. Lo oigo a través de los audífonos del helicóptero.

\reply Sí, cáncer. ¿Sabes lo que es el cáncer exactamente? Solo una célula que pierde la capacidad de envejecer y morir. Se clona y... Sigue ahí. Y se clona con ese \emph{defecto}. Así una y otra vez. Nos mata porque satura y desestabiliza nuestro organismo, pero, en esencia, es la supervivencia en estado puro. Se trabajó mucho en cómo eliminarlo, pero no en aprender de él.

\reply ¿Y consiguieron algo? \pause pregunto.

\reply No puedo contarte demasiado, pero si, se consiguieron cosas. Solo sé lo que me llegaba. Nunca he estado en las instalaciones. Allí solo queremos ver resultados tangibles.

Miro con una sonrisa a mi compañera de al lado, Yuva, que también estaba pendiente a la conversación.

\reply ¿Qué te parece? La inmortalidad \pause le digo incrédulo.

Algunos miembros del equipo nos miran. Klave sonríe y dice:

\reply ¿Podrían pagarnos con algunas dosis de ese cáncer mágico, Ethom?

\reply Cuando lo consigamos va a costar mucho dinero, Klave. Le aconsejo ahorrar desde ahora e invertir parte de su sueldo en acciones de la compañía \pause replica Ethom.

El helicóptero se estremece y todos nos tensamos en nuestros asientos. El piloto nos habla por los audífonos:

\reply Estamos llegando. En un minuto estamos en tierra.

Muevo la mano derecha en el aire formando círculos.

\reply ¡Vamos chicos! A partir de este momento entramos en operativo. Repaso de factores. Peligro biológico, Klave.

\reply Traje NNbq activado y cargado. Lanzallamas purificador operativo y cargado, señor.

\reply Resistencia con armas de fuego en interior, Nivia.

\reply Armas cargadas y limpias. Tres cargadores rifle. Dos pistolas. Granada cegadora. Granada de fragmentación y botiquín personal, señor.

\reply Combate cuerpo a cuerpo, Yuva.

\reply Protectores predator y cuchillo táctico, señor.

``Qué hermosa es. Hermosa, sensual y letal. Mi niña asesina.''

\reply Venga, todos abajo.

Salimos todos en formación. La zona está iluminada y tranquila. Ethom va entre nosotros, agachado y cargando un maletín. El suelo está parcialmente congelado y hay restos de nieve alrededor.
Ethom nos señala la entrada y avanzamos mientras el helicóptero despega, llevándose el atronador sonido. Al poco tiempo solo se escucha el sonido controlado de nuestros pasos. Al llegar a la puerta me giro y compruebo al equipo.

``Yuva. Klave. Nivia. Doll. Todos pertrechados y atentos.''

\reply Bien, Ethom. Repasemos. A partir de ahora tú solo indicas la dirección a seguir con el dedo y no abres la boca. Te mantienes siempre tras Yuva. ¿De acuerdo?

Me hace un gesto con el dedo hacia la puerta. Al grano. La puerta es pequeña. Parece una garita de meteorología o vete a saber qué. Algo perdido en mitad del yermo congelado. En realidad es una entrada secundaria a las instalaciones. Abrimos y entramos.

Nuestra principal preocupación es que un grupo armado haya secuestrado las instalaciones, así que comprobamos minuciosamente cada ángulo. Dentro esta oscuro, húmedo y silencioso. A medida que bajamos, la humedad se hace mas notoria y caliente. Lo noto en la cara. 
Ethom va indicándonos en silencio. Pasillo, escalera, pasillo, sala de reuniones\textellipsis\ Hasta que llegamos a lo que parece una sala de seguridad. Cuando cierra la puerta, Ethom rompe el silencio.

\reply Sala hermética, no se oirá nada fuera.

De pronto, un sonido potente como de rasgadura surge de una esquina. Cuando miramos, Klave tiene expresión de alivio y mira al techo suspirando.

\reply Joder, qué ganas tenía.

\reply Eres un puto asqueroso, Klave \pause dice Nivia.

El olor nos inunda rápidamente mientras empiezo a visualizar las cámaras de seguridad.

\reply ¿Otra vez estofado de tu madre, Klave?

El hijoputa de ríe. Ya nos tiene acostumbrados. Ethom está trabajando frente a un grupo de pantallas con datos bastante complejos.

\reply ¿Saca algo en claro, doctor?

\reply Las mediciones no están mal del todo.

\reply Explíquese.

\reply Bueno, los niveles de oxígeno están un poco bajos. No detecto actividad humana. Nadie está usando nada ahora mismo. Es posible que haya sido un ataque biológico. ¿En las cámaras no se ve nada?

\reply No. Por ahora no veo a nadie. ¿Qué porcentaje de instalación es visible desde estas cámaras?

\reply Mm, supongo que un diez o un quince por ciento.

\reply Bien. Haremos una hora de observación y toma de datos desde aquí. Klave y Nivia salid a comprobar los alrededores. Y dejad la puerta abierta.

Me siento y me dispongo a escudriñar cada centímetro de instalación mediante esas cámaras. ``Un diez por ciento. Tenemos un noventa por ciento de sorpresas.'' Doll, el gigante ruso con cara de adolescente, está conectado a la sección de transporte. Se supone que este puesto está conectado a las instalaciones principales, a unos diecisiete kilómetros, por una red de vías por las que circulan pequeños vagones. Vamos a ir en uno de los que se usan para las visitas, mas lujosos y, lo mas importante, mas silenciosos que los de carga de personal o material.
\reply Capi \pause me llama Doll\pauseend. Mire lo que he conseguido.

Al acercarme a su pantalla veo la imagen de una cámara diferente a las que he estado mirando.

\reply Es del vagón que viene a recogernos. Lo estoy moviendo a la mínima velocidad para no provocar ruidos. Está en la misma entrada de las instalaciones. Vi algo raro y lo paré. Mire. 

Me señala con el dedo algo en la pantalla. Al principio no veo nada, pero después, algo se mueve. Sin querer bajamos el tono de voz como si pudiera oírnos. Hay un bulto en el muelle de carga. Parece una persona durmiendo que convulsiona de vez en cuando. Solo le vemos la espalda y con muy poca definición.

\reply ¿No tiene zoom, verdad?

\reply No, señor.

De súbito, la cámara se estremece y al instante unos pies aparecen frente a la cámara. Parece que ese hombre hubiera saltado desde el vagón, que ahora está quieto.

\reply Nos han detec\textellipsis\ \pause empiezo a decir. 

Pero pasa algo raro. El hombre, con la ropa manchada de rojo y bastante rota, corre hacia el que está durmiendo, se le echa encima y empiezan a pelear. Lo hacen de una forma extraña. Se miran, se agarran de la ropa y parece que se gritan.

\reply ¿Tienes audio, Doll?
\reply Estoy en ello. El vagón tiene un micrófono para el guía y lo\textellipsis\  Tengo\textellipsis\  ¡Ahora!

Por los altavoces se escuchan unos aullidos desquiciados. Gritos que no se sofocan. Un grito de dolor terrorífico mantenido en el tiempo de manera antinatural.

\reply ¿Qué coño?

Justo llega Klave y Nivia para escuchar el espectáculo. Todos miramos a la pantalla con expresión preocupada. Aquellos dos se están revolviendo, dando gritos y agarrándose como auténticos locos. Me recuerda a cuando tuve que patrullar las calles y veía a gente desquiciada por haber tomado krokodile, una droga que te volvía loco. Al poco tiempo, los dos se quedan quietos. Uno encima del otro. Ethom dice con voz entrecortada:

\reply Es\textellipsis\ Es de la com\textellipsis\ Compañía. Su\textellipsis\ Traje.

El que está debajo lleva puesto un mono azul con una raya blanca, bastante sucio y con partes hechas jirones.

\reply ¿Alguien tiene alguna idea? \pause pregunto a los demás.

\reply Ataque biológico con algún fármaco que altera\textellipsis\ Bueno, eso que vi\-mos \pause contesta Yuva.

\reply Posesión demoníaca \pause bromea Klave.

\reply Locura\textellipsis\ Mm, transitoria por un hecho traumático \pause apunta Ethom.

\reply Apuesto por el doctor Ethom, pero iremos con el NNbq al máximo de protección \pause digo\pauseend. Doll, trae ese trasto rápido.

Poco después estamos en el muelle de carga esperando a que llegue el vagón. Ya se ve al fondo. Un punto de luz entre la penumbra, que se convierte en un gran bulto blanquecino y\textellipsis\ Rojo. Cuando llega y de detiene, nos quedamos atónitos unos segundos. El vagón está ensangrentado, sobre todo por fuera. Hay pequeños pedazos de carne pegados aquí y allá. El techo, de donde bajó aquel tío, es lo peor. Parece que hubieran matado a una res ahí encima. Los restos caen lentamente al interior y por los lados. Menos mal que tenemos los trajes.

\reply Todos adentro.

Doll maneja el vagón y lo redirige de vuelta a la estación. Va a treinta kilómetros por hora, así que tenemos media hora por delante. Yuva toca los restos que caen del techo con los guantes.

\reply Estos restos son extraños. No sé bien a qué parte del cuerpo pertenecen, pero en general parecen demasiado viscosos. Para mí está claro que es un agente químico que causa esto, además de un desequilibrio emocional extremo.

Me acerco y toco los restos de su mano enguantada con la mía, pensando al instante en el tacto de sus dedos bajo el plástico.

\reply Podría ser. A partir de ahora, ante cualquier contacto hostil, tiraremos a matar.

Todos me miran decididos, hasta Ethom, que ha traído su propia pistola y, según él, es bueno usándola. El sonido que produce el vagón al moverse es monótono y agotador. ``Un momento. ¿Agotador? ¿Qué es ese otro sonido?''

\reply ¿Alguien más escucha ese sonido?

Me miran sin comprender.

\reply Doll, detén el vagón.

Escucho algo además del monótono sonido del vagón. Un sonido agotador. ¿Cómo es posible que haya pensado en un sonido agotador? No tiene sentido. Mi mente se estaba cansando de escuchar ese sonido oculto tras el del vagón. ¿Habría alguna fuga en mi traje? El vagón se detiene lentamente y a medida que lo hace, el otro sonido se hace mas notable. Klave frunce el ceño tras la máscara.

\reply Sí, escucho eso también. Es como un zumbido y un gorgoteo muy bajo. Entiendo por qué te preocupa.

Los demás me miran y asienten. Empezamos a buscar el origen del sonido. Yo empiezo a ubicarlo cerca del techo. ¿Un emisor de frecuencias enajenadoras? He oído hablar de eso. Yuva sale del vagón y se encarama al techo. Al momento escucho como grita y salta del techo hacia las vías, casi tres metros mas abajo.

\reply ¡Yuva!

Está tendida en el suelo boca arriba. La ayudo a incorporarse y toco sus piernas. Parece que todo está bien. Pero ella solo señala al techo y dice ``eso es imposible''. 

Dejo a los demás con ella y me encaramo poco a poco al techo. Al llegar a la altura veo la masa de carne gelatinosa que se escurría poco a poco al interior. Subo un poco mas y busco eso que asustó a Yuva. Aquí el sonido se escucha mas, está claro. Además de escucharlo por los oídos, se te mete en la mente. Un momento, veo algo. ¿Qué demonios es eso? Por un instante, las piernas se me aflojan y las noto frías. Quiero saltar a las vías pero me aferro con las manos a un reborde del techo con toda mi voluntad. Tras el impacto inicial, me recupero, pero no quiero acercarme a esa cosa. Apuntándole con mi rifle, ajusto la mirilla a varios aumentos y enciendo la linterna. Apunto al montón de carne, a un punto negro que borbotea en ella. Por la mirilla veo algo imposible; una boca en miniatura, como de un centímetro de ancho y repleta de pequeños dientes. Se mueve y vibra, exactamente al mismo compás que aquel extraño sonido. Está claro que ``eso'' es lo que lo produce. Justo encima de la boca hay una esfera brillante y rosada encajada en la carne. La esfera se mueve y tiene una serie de puntos mas oscuros que se dilatan cuando le apunto con la linterna. Un ojo.

Algo afectado, hago subir uno a uno a los miembros del equipo para una exposición controlada. Todos necesitamos saber a qué nos estamos enfrentando. Luego, activo el pequeño lanzallamas acoplado en el rifle y quemo a la cosa. Automáticamente el zumbido aumenta de intensidad, dejándonos ``sordos'' unos instantes y luego desaparece. Ahora solo se escucha el crepitar de las llamas.

\reply Dios, qué alivio \pause dice Klave\pauseend. Era mucho mas intenso de lo que parecía.

Todos asentimos. En efecto, ahora que ya no está, nos damos cuenta de que era mucho mas penetrante de lo que percibíamos.

\reply ¿Qué pensamos sobre esto? \pause pregunto.

\reply Aquí se investiga más que eso del cáncer, ¿eh doctor?. \klave

\reply No tengo constancia de ello. Pero parece que sí. Creo que estamos seguros de que no fue una alucinación, aunque deberíamos apagar el fuego y volver a verlo.

\reply En el vagón hay un extintor, voy a por él. \yuva

Poco después todos miramos una masa chamuscada en el suelo de las vías. El doctor está agachado y con un cuchillo corta pedazos de la masa buscando la boca. Al poco tiempo da con algo y nos lo enseña en la punta del arma.

\reply Sí, aquí está \pause dice \pauseend. Es una estructura cartilaginosa rodeada de\textellipsis\ Tumores.

\reply ¿Pero cómo es posible? \klave

\reply No lo sé. Se me pasan muchas cosas por la cabeza, pero ninguna con sentido.

\reply Bueno. Sigamos. Extrememos las precauciones. Atentos a posibles zumbidos \pause ordeno.

Montamos otra vez en el vagón. Ordeno a Doll que lo pare unos doscientos metros antes del muelle de carga. Avanzamos con mucho cuidado entre las vías. A medida que llegamos, veo restos de sangre y algún pingajo de carne sueltos, pero no noto ningún zumbido.
Al llegar, señalo con el arma a los dos cuerpos que siguen juntos. Ahora se ve claramente que los cuerpos están ensangrentados y demacrados.

\reply Señor. Levántese con cuidado y ponga las manos sobre la ca\-be\-za \pause gri\-to.

El que está encima da un respingo, apoya el brazo derecho (cubierto de llagas y deforme) en el suelo y empieza a incorporarse. Inmediatamente empiezo a escuchar un zumbido parecido al de antes, que se me mete en el cerebro y me nubla los sentidos. El hombre termina de levantarse y no es un hombre. Es horrible. Su imagen se desenfoca mientras intento apuntarle con el arma. Están los dos pegados. Sujetos por jirones y tiras de carne. Sus cuerpos están abombados y empiezan a salir brazos de debajo de sus ropas. Muchos brazos. Donde debería estar su rostro hay una espiral de hueso, carne y ojos. Entre sus dientes repartidos por toda la cabeza salen mórbidas lenguas afiladas y viscosas. Las bocas se abren y gritan, y el chirriar en mis oídos me hace hincar una rodilla en el suelo. Consigo apretar el gatillo del rifle y una ráfaga de balas alcanza al ser. La presión en los oídos disminuye de manera drástica y veo que los miembros de mi equipo se recomponen y disparan al engendro, que cae entre una lluvia de pedazos de carne. El zumbido, que hace colapsar el cerebro, pierde intensidad hasta casi desaparecer. Luego, al quemarlo, desaparece del todo. Nuevamente nos quedamos observando cómo arden los restos durante un rato.

\reply ¿Todos bien?

\reply Dentro de lo que cabe.

\reply Vamos a dividirnos en dos equipos. Yo, Doll y Ethom delante. Yuva, Klave y Nivia, detrás a unos treinta metros. Encendemos comunicación. Si nos veis flaquear, acabad con cualquier cosa que tengamos por delante. Espero que esa influencia que son capaces de irradiar no os afecte a esa distancia.

Así avanzamos por la red de pasillos guiados por Ethom. Éste consulta un mapa bastante complejo a medida que nos movemos. No vemos mas de esas cosas, aunque sí vemos restos de carne que enseguida son chamuscados por nuestros lanzallamas. En una zona despejada, una especie de comedor, me detengo para reunir al equipo. Los tres de atrás, los mejores tiradores, están a unos veinte metros cuando una puerta tras ellos se abre de golpe con un estruendo y del hueco brota una cantidad ingente de esos seres. Veo a Yuva, Klave y Nivia huir hacia nosotros, pero las piernas les fallan. Abrimos fuego y los seres de atrás, esos malditos zumbadores, empiezan a caer. Cada tiro es una décima de segundo de recuperación para los tres que huyen. ``Vamos, vamos''. Al final consiguen zafarse del radio de acción de esos zumbadores. En cuanto se alejan unos ocho metros empiezan a correr con normalidad. Yuva desenrosca una granada de fragmentación y la manda hacia atrás. Cae entre la masa de cuerpos y es engullida al instante.

Todos corremos intuyendo el rumbo. Los seres son demasiados. Afortunadamente no son muy rápidos. Cuando la granada explota, la masa de cuerpos se convulsiona y algunos pedazos de carne saltan por encima. Puedo ver la estructura de la masa. A pesar de ser un montón de cuerpos, muchos están unidos entre si, como si dedos de una enorme mano se tratase.
Le grito al doctor que nos encuentre una salida mientras corremos entre pasillos y puertas metálicas.

\reply La sala de investigación. Está cerca y está aislada.

Es justo a donde tenemos que ir así que perfecto. Hemos perdido a la masa de vista pero seguimos corriendo. Nos detenemos un segundo en una bifurcación para que el doctor se aclare cuando escuchamos de nuevo el estruendo, esta vez delante de nosotros. Las puertas de doble hoja que tenemos enfrente se doblan y se rompen bajo el peso de otra masa de cuerpos. De nuevo el zumbido otra vez. Agarro a alguien de la mano y empiezo a correr. ``Solo diez metros''. El pasillo frente a mí da vueltas y vueltas. Miro a quién tengo agarrado. Es Yuva. Mas atrás no hay nadie. Solo la masa de cuerpos. Ver a Yuva me da fuerzas y consigo la ventaja necesaria. A partir de los diez metros su intensidad cae muy rápido y podemos correr.

\reply ¿Chicos, dónde estáis? \pause pregunto por el comunicador sin dejar de correr.

\reply Llegamos a la sala de investigación \pause responde Klave\pauseend. Estamos el doctor, Doll y yo. ¿Los demás están con usted?

\reply Estoy solo con Yuva. No sé nada de Nivia.

\reply ¡Joder\textellipsis! Bueno, dice el doctor que vaya hacia el gimnasio, es un módulo aparte que se puede sellar. Nivia, si lo escuchas dirígete hacia allí.

Sigo a trote los carteles de los pasillos hasta encontrar el gimnasio. Aún escucho detrás, a lo lejos, a la masa avanzando. Entro con Yuva en el gimnasio, cierro la puerta y el sonido de los retenes de seguridad de ésta me permite descansar un momento.

\reply Recopilad toda la información e informad con lo que sea.

Luego me giro hacia Yuva y la abrazo a través de los trajes. Nuestras máscaras se tocan y ella me mira con sus preciosos ojos. Algo se revuelve dentro de mí. Siento que va a ser la última vez que la veo. Tengo ganas de quitarme la máscara y besarla. Algo mareado por el excesivo consumo de oxígeno veo como Yuva casi me arranca la máscara, se quita la suya y me besa intensamente. El corazón me late fuerte y rápido. La aprieto contra mí. La erección es casi instantánea y amenaza con atravesar el traje. Me lo quito y ella se quita el suyo. En un segundo estamos completamente desnudos. Me arrastra a una camilla de masaje y me rodea la cintura con sus piernas. Sus muslos están ardiendo de la actividad y la excitación. La penetro casi sin querer. Sus ojos se van hacia atrás. Es maravilloso. La embisto hasta que nos hacemos uno\textellipsis