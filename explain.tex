Este documento está generado con \LaTeX, un software de composición de textos orientado a la creación de documentos escritos que presenten una alta calidad tipográfica. Utiliza un lenguaje de composición tipográfica denominado \TeX, y está especialmente orientado a la redacción de documentos científicos.

Sin embargo, también se puede utilizar para redactar una obra de ficción como ésta. En Internet hay multitud de plantillas que se pueden utilizar para generar libros. Todas las secciones, capítulos, tablas de contenidos, imágenes, etc. que veis en el documento son configurables de una forma muy flexible.

Por ejemplo, en esta página se pueden poner las dedicatorias. Pero si se decide que no hay dedicatorias, pues se borra esta referencia del código fuente, y se genera un libro sin esta página. Y lo mismo con el resto de cosas que veáis en el libro. La flexibilidad de \LaTeX\ es practicamente infinita, y es una herramienta muy potente. Software libre, y filosofía \emph{Do it yourself} total.