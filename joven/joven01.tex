Nunca pensó que pudiesen llegar a fabricar una réplica tan perfecta. La curva de sus labios, su delicada tez, el cabello rodeándole el cuello. Todo era una reproducción exacta de lo que había sido Mariela. De su Mariela de hacía cincuenta años. A través de la pequeña abertura acristalada que había en el embalaje se podía contemplar su reconocible rostro. Estaba todavía dormida, ajena a lo que ocurría en el exterior. Mirándola allí dentro, Walter comenzó a experimentar una emoción que le resultaba muy familiar. Tuvo la impresión de viajar de golpe al pasado, a una época en la que Mariela y él tenían un infinito futuro por delante. Una época en la que hubiese sido inconcebible pensar en algo remotamente parecido a lo que ahora estaba ocurriendo en el salón de su casa.

Era cierto que la tecnología había avanzado más de lo que cabía esperar desde que Walter realizara el encargo. O, al menos, eso pensaba alguien como él, un neófito en cuestiones tecnológicas. Se había pasado media vida viendo aburridas señales de vídeo en los monitores de un sistema de seguridad perimetral. Una tecnología nada sofisticada, a decir verdad. Y cuando esa labor de vigilancia pasó a ser desempeñada casi en su totalidad por máquinas más modernas, que tenían la capacidad de analizar en tiempo real cientos de señales de vídeo simultáneas, apenas tuvo que interactuar con la nueva tecnología. Su trabajo se vio reducido a comprobar las falsas alarmas que muy de vez en cuando emitía el sistema automático de vigilancia.

No fue hasta hacía siete años cuando empezó a interesarse por los llamativos productos que comercializaba la compañía Synteesi-Models, la empresa que había conseguido lanzar al mercado los modelos de androide más eficaces y realistas que se habían visto nunca. Cada nueva versión que ponían a la venta generaba una expectación cada vez mayor. Se servían de vistosas campañas publicitarias, como colocar algunos de sus modelos dando vida a personajes en obras de teatro de primer nivel, o haciendo demostraciones en los programas de entretenimiento televisivo de mayor audiencia. En una ocasión consiguieron reemplazar al completo la tripulación de cabina de un avión por varios de sus modelos de última generación. El vuelo se desarrolló con normalidad, con las máquinas humanoides realizando las labores de asistentes durante todo el trayecto. Y solo tras aterrizar se anunció a los pasajeros que habían sido atendidos por modelos sintéticos en lugar de personas. Algún que otro pasajero llegó a interponer una queja. Pero en el balance final, el impacto mediático alcanzado con esa acción publicitaria había reportado a Synteesi-Models un crédito casi inagotable de prestigio, visibilidad y, a la postre, pingües beneficios.

Y aún así, pese a toda la buena fama que la compañía había llegado a acumular, el nuevo producto que Walter acababa de recibir superaba con creces sus expectativas más elevadas. Tras el primer vistazo, tuvo la impresión de que todo el dinero invertido había valido la pena. Y eso que todavía no la habían sacado del embalaje.
	
	\reply Tengo que darle las gracias, Walter.

La voz sonó a su espalda, grave y rotunda. Walter se dio la vuelta y encontró la mano tendida de Nikola Virtanen, el presidente de Synteesi-Models.
	
	\reply ¿A mí? \pause dijo Walter mientras estrechaba la mano a Virtanen\pauseend. Yo pensaba felicitarlos a ustedes por su impactante trabajo.
	
	\reply Gracias \pause respondió Virtanen, haciendo una leve reverencia con la cabeza mientras exhibía una sonrisa de satisfacción\pauseend. Pero es la fe de usuarios pioneros como usted la que nos motiva a superar nuestras propias cotas de excelencia. Sin embargo, Walter, debo confesarle que el listón que hemos puesto con esta primera unidad del modelo Syntynyt-Uno no va ser fácil de superar. Usted es desde hoy el primer usuario de Syntynyt-Uno de la historia y queremos que esté lo más satisfecho posible con nuestra creación. Le aseguro que se trata de un importantísimo hito para nuestra compañía, y por nada del mundo habría querido perderme algo así.

Dos técnicos de Synteesi-Models estaban realizando las labores de desembalaje. Mariela se encontraba en el interior de una suerte de cápsula alargada, una estilizada caja a la que le habían instalado unos soportes auxiliares para realizar el transporte desde la fábrica. Uno de los técnicos había conectado un ordenador portátil a una toma que había en el exterior de la cápsula, y estudiaba con atención las diferentes lecturas que se iban mostrando en la pantalla. El otro técnico estaba retirando los precintos y sujeciones que envolvían la cápsula, preparándola para su apertura.
	\reply En cuanto los sistemas de su Syntynyt-Uno asimilen la atmósfera de la casa, procederemos a su activación \pause explicó Virtanen\pauseend. Mientras tanto, le pediría que me acompañase a la mesa para terminar de formalizar la entrega.
	
Walter acompañó a Virtanen hasta la gran mesa de su salón sin dejar de prestar atención al trabajo de los técnicos. En la mesa había un voluminoso maletín abierto. Contenía diversos artículos que formaban parte del paquete de entrega del Syntynyt-Uno. Sobre la mesa también había varios documentos dispuestos de forma meticulosa. Virtanen ofreció una pluma a Walter.
	
	\reply Solo necesitamos unas cuantas firmas suyas que certifiquen la correcta recepción de la entrega \pause dijo Virtanen mientras Walter cogía la pluma que le había ofrecido\pauseend. En el maletín encontrará tanto los diversos manuales de usuario como el documento completo de garantía y condiciones de uso. Simplemente le recordaré que el Syntynyt-Uno está diseñado para uso doméstico exclusivamente, al menos en esta primera versión. Le recomendamos, por tanto, que no haga uso de él en el exterior de la casa. Puede tener la seguridad de que seguiremos trabajando para mejorar las características de nuestros productos. Pero en este momento debo recalcarle la importancia de esta cuestión. También encontrará en el maletín todo lo necesario para la alimentación del Syntynyt-Uno \pause Walter firmaba los papeles impacientemente mientras Virtanen continuaba hablando\pauseend. Este modelo debe tomar dos raciones diarias de Synty-Soylent. Su preparación es muy sencilla, se trata de un preparado soluble en agua que contiene todo el aporte energético que necesita su modelo. Además de esto, el modelo debe dormir una vez por semana en la cápsula en la que lo hemos transportado hasta aquí. Es necesario para recargar la energía de algunos de sus sistemas internos. Se hace a través de un sistema inalámbrico que incorpora la propia cápsula. Pero como le digo, solo será necesario una única noche por semana.
	
Walter estampó la última firma y devolvió la pluma. Virtanen tomó aire y echó un vistazo a la cápsula. Los técnicos ya casi habían terminado con los preparativos.
	
	\reply Creo que no se me olvida nada, Walter \pause dijo Virtanen, mientras recogía los documentos firmados\pauseend. De todas maneras, como sabe, hay un número de teléfono de atención exclusiva a su disposición. Puede usarlo para hacer cualquier consulta. Y a la hora que quiera.
	
Virtanen sacó una tarjeta de uno de sus bolsillos y se la entregó a Walter, que la guardó sin siquiera mirarla. En ese momento no deseaba anticipar ningún contratiempo futuro. Solo quería estar con Mariela. El sonido de un sistema de apertura neumático calmó su creciente ansiedad.

La compuerta de la cápsula se abrió. Walter y Virtanen se aproximaron con expectación. Unos dedos delicados se asomaron desde el interior. La mano a la que pertenecían se apoyó en el borde de la compuerta. A continuación, el Syntynyt-Uno se incorporó por completo y salió de la cápsula. Un leve escalofrío pareció recorrer su cuerpo cuando sus pies desnudos tocaron el suelo de la sala. Una gran túnica era lo único que cubría su cuerpo. Miró a su alrededor, recorriendo la estancia con la vista, entre sorprendida y curiosa. Virtanen dio un paso adelante para hablarle.

\reply Hola. Mi nombre es Nikola, y quiero que sepas que es un placer conocerte.

Virtanen le tendió la mano. Ella, con un tímido movimiento, le estrechó la mano y le dedicó una cándida sonrisa.

\reply Quiero presentarte a Walter \pause continuó Virtanen, dando un paso lateral para enfatizar la presencia de Walter\pauseend. Estás aquí gracias a él. Y a partir de hoy, este será tu hogar.

El Syntynyt-Uno miró a Walter y le sonrió en señal de gratitud. Walter permanecía en su sitio, sin reaccionar todavía. Los técnicos terminaron de recoger sus equipos y se dirigieron a la salida. 

\reply Les dejamos a solas, Walter. Y recuerde, quedamos a su disposición.

Tras despedirse, Virtanen siguió los pasos de los técnicos y abandonó con ellos la casa. Walter no respondió. Todavía inmóvil, sosteniéndose la mirada con el Syntynyt-Uno, se preguntaba qué pensamientos, o cálculos, estarían teniendo lugar en ese momento tras esos ojos que lo miraban con reeditada curiosidad. Eran sin lugar a duda los ojos de Mariela. Ella se acercó hasta hacer desaparecer la distancia entre ambos. Tenía la estatura exacta de Mariela, con la punta de su nariz tocaba el mentón de Walter.

\reply Walter\textellipsis

Su voz. Era su voz la que resonaba nítida en la cabeza de Walter, pese a ser el eco de un tiempo distante. Walter introdujo sus manos bajo la túnica que cubría a Mariela, y su sentido del tacto le recordó la geometría específica de su cuerpo. La túnica se deslizó poco a poco hasta el suelo.

\reply Mariela\textellipsis\ \pause susurró Walter.

La única nota discordante era un pequeño tatuaje de color verde, con el logotipo de la compañía, en el brazo izquierdo. Algo de lo que la Mariela original carecía. En cuanto a todo lo demás, era la materialización precisa de un anhelo cumplido. Walter acercó su cara a la de Mariela. Su olor lo envolvió en una embriagadora nebulosa. Y cuando la besó, Walter sintió que rejuvenecía cincuenta años en el acto.