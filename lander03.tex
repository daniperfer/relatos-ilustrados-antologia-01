La oleada de anfetaminas estalló sobre los párpados de Pol, bajando como una cascada, un cosquilleo químico que alteró gradualmente el ritmo cardiaco. La sensación de cansancio se dispersó, disuelta en una agradable marea de vigor. La inyección proporcionada por los sistemas programados del traje hizo que se olvidara del agotamiento y reforzó su concentración. El ascenso por la montaña de escombros empezó a parecerle menos pesado y se lo tomó como un juego de resistencia. Se descubrió a sí mismo ganando altura a grandes zancadas, con un aire jovial, poniendo todo el cuidado en no pisar los fragmentos que rodaban pendiente abajo, desplazados por las pisadas de Frida. La llana extensión de terreno calcinado, con el telón del atardecer velado y rojizo, se le antojó la visión de un diorama. Una vez que coronaron la colina de cemento y acero, Frida se paró, recta y vigilante, encaramada a los restos de una cúpula que aún conservaba su recubrimiento de cal. Después de utilizar su visor, realizó el descenso a base de largos saltos, provocando al caer pequeños aludes de piedras que sonaron como huesos. Pol la seguía, atrapado en la ilusión de que Frida, tan atlética, era una persona de bronce que había escapado de su pedestal. La inercia de la bajada provocó que Pol tuviera que frenar golpeando el suelo con una sucesión de patosas pisadas. Se detuvo frente a Frida. Dos soles helados ardían poniéndose tras el flequillo de su compañera. ¿Qué había en esa mirada? Frida parecía tan maternal como recia. Estaba seguro de que bajo su presencia árida corría un manantial de dulzura, que tal vez fuera solo un hilillo. Sintió el irrefrenable deseo de tocarla, pero un gesto seco le sirvió para recomponerse y recuperó el paso. Había que atravesar el llano lo más rápido posible; el calor y la exposición a la radiación podían llegar a ser insoportables y no tendrían el resguardo de las ruinas. 