Sed bienvenidos a este relato cuyo nombre vais a recordar. Consideradme vuestra fiel guía en este viaje, pensad que cada vez que bajéis la cabeza y leáis mis palabras alguien os estará observando, y os fijaréis en las luces, en las sombras, miraréis de un lado a otro, buscando respuestas o rostros que en realidad esperáis no encontrar.  Os advierto, esto no es una broma.  Es scopaesthesia. 

¿Nunca os habéis sentido como si todos los ojos se posaran en vuestra persona, como si hasta los árboles abrieran los ojos y os observaran agresivamente y sin disimulo? Y entonces, casi sin daros cuenta, ¿os hicierais pequeños, del tamaño de una hormiga evitando así las miradas, escondiendo vuestro cuerpecillo detrás de una hoja? Pues a mí me pasa todos los días y os aseguro que aquí no hay pócimas que me hagan crecer, esto no es el país de las maravillas, esto es el país de los maravillados y hace tiempo que alguien mató al conejo. 

¿Estoy loca? ¿No estoy loca? Qué más da, la verdad nunca fue tierra firme, solo arenas movedizas. Dije que os iba a guiar, y eso voy a hacer pero antes que nada tengo que advertiros que todo lo que voy a contaros es real y tengo pruebas.  Para que tengáis una mínima noción de lo que os intento transmitir voy a ser lo más generosa posible y empezaré por el principio. 

Llevo meses, qué digo meses, años, viviendo con este secreto,  vivir como eufemismo porque esto ha sido y es un suplicio. Justo ayer descubrí que era lo que me estaba atormentando, lo que lleva tanto tiempo provocándome esta ansiedad y sigo sin poder ni siquiera nombrarlo, no soy capaz de asimilarlo y por eso debo contarlo, sacar a la luz estas sombras ante vosotros, amados desconocidos.