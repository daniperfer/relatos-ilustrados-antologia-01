Sed bienvenidos a este relato cuyo nombre vais a recordar. Consideradme vuestra fiel guía en este viaje, pensad que cada vez que bajéis la cabeza y leáis mis palabras alguien os estará observando, y os fijaréis en las luces, en las sombras, miraréis de un lado a otro, buscando respuestas o rostros que en realidad esperáis no encontrar.  Os advierto, esto no es una broma.  Es scopaesthesia. 

¿Nunca os habéis sentido como si todos los ojos se posaran en vuestra persona, como si hasta los árboles abrieran los ojos y os observaran agresivamente y sin disimulo? Y entonces, casi sin daros cuenta, ¿os hicierais pequeños, del tamaño de una hormiga evitando así las miradas, escondiendo vuestro cuerpecillo detrás de una hoja? Pues a mí me pasa todos los días y os aseguro que aquí no hay pócimas que me hagan crecer, esto no es el país de las maravillas, esto es el país de los maravillados y hace tiempo que alguien mató al conejo. 

¿Estoy loca? ¿No estoy loca? Qué más da, la verdad nunca fue tierra firme, solo arenas movedizas. Dije que os iba a guiar, y eso voy a hacer pero antes que nada tengo que advertiros que todo lo que voy a contaros es real y tengo pruebas.  Para que tengáis una mínima noción de lo que os intento transmitir voy a ser lo más generosa posible y empezaré por el principio. 

Llevo meses, qué digo meses, años, viviendo con este secreto,  vivir como eufemismo porque esto ha sido y es un suplicio. Justo ayer descubrí que era lo que me estaba atormentando, lo que lleva tanto tiempo provocándome esta ansiedad y sigo sin poder ni siquiera nombrarlo, no soy capaz de asimilarlo y por eso debo contarlo, sacar a la luz estas sombras ante vosotros, amados desconocidos.  

No puedo recordar cuando empezó, solo sé que una mañana cualquiera, no fue el despertador quien me arrancó de los brazos de Morfeo, de repente noté una presencia y mis ojos se abrieron de par en par como platos, sentí que me faltaba el aire, algo colapsaba mis pulmones, mi corazón se aceleró como si estuviese ante un peligro de muerte inminente, sentía como la sangre circulaba por todas y cada una de mis arterias, un sudor frío se apoderó de mi cuerpo y noté una leve presión en mi cuello. Pensé que seguramente había sido una pesadilla, no le di ninguna importancia, y es que a veces suelo ser ingenua, desdichadamente ingenua. Bajé mis pies sudorosos y los puse en contacto con el suelo helado de color verde pistacho. Me quedé embobada mirando como los dedos de mis pies se movían hacia arriba y hacia abajo  impacientes por el nuevo día que les esperaba. Dejé de lado mi experiencia onírica y me dispuse afrontar mi largo día de jornada laboral, y como cada día lo primero y esencial era la banda sonora que me acompañaría en mis aventuras corrientes. Busqué el botón detrás de mi oreja y sintonicé Skinny Love con un leve toque detrás del pabellón auricular que conectaba directamente con mi oído interno, entonces empezó a sonar lo que marcaría el inicio de mi día : ``I told you to be patient, I told you to be fine, I told you to be balanced…'' Me reconfortaban sus palabras, su voz y esa guitarra. En cuanto tuve la música sintonizada me puse en marcha.  Mi pelo seguía mojado y llevaba las gafas medio empañadas, había terminado mi taza de té e iba corriendo como de costumbre porque llegaba estrepitosamente tarde.  Por suerte el superloop que pasaba a las ocho y media aún no había llegado, amaba montarme en el transporte público y observar a través de los cristales la panorámica de la ciudad, todavía me asombraba como esos tubos transparentes recorrían de arriba abajo y de izquierda a derecha mi amada metrópoli. Me senté a esperar mientras observaba a los extraños cuyo rostro olvidaría a los cinco segundos de cruzarnos las miradas, la mayoría iban conectados, yo les llamaba los desfrontalizados, eran aquellos que habían decidido formar parte de la mente colmena gozando así del privilegio de tener la mayor parte de las funciones corporales automatizadas mientras su mente navegaba en mundos virtuales donde podían ser quiénes nunca serían en el mundo real. Bienvenida mediocridad ponte cómoda, vamos a acostarnos, y te aviso, vas a gemir. Conformismo, comodidad, miedo o todos ellos envueltos y revueltos.  Yo no iba a juzgar a nadie, pero si que me permitía satisfacer mis necesidades artísticas dibujando las facciones de aquellos entes tan peculiares y totalmente absortos en sus mundos imaginarios. Cogí asiento, al lado de la ventana, como siempre, saqué mi cuaderno y empecé a buscar entre el gentío un rostro especial, una verdad, un destello. Y de pronto comenzó todo, esta pesadilla en la que me veo atrapada, sentí como alguien me observaba, tenía la sensación en lo más profundo de mi ser que alguien estaba postrando sus ojos sobre mí, me giré y había un anciano leyendo un libro. No quise insistir y pensé que habían sido imaginaciones mías, pero probablemente si hubiese puesto más atención, si le hubiese mirado directamente a los ojos, no estaría ahora mismo hasta el cuello. El día no hacía más que enrarecerse, pero quién era yo para sospechar nada, normalmente vivo en días raros. En mí estaba comenzando a crecer como si de una enredadera se tratara la ansiedad y la angustia. Me bajé del superloop en una parada cuya localización desconocía, me había puesto tan nerviosa que ni siquiera me había dado cuenta de que estaba en la zona más conflictiva de la ciudad. Por suerte era de día, no quiero ni imaginarme que hubiese pasado si me hubiese encontrado en ese recóndito lugar a las doce de la noche. Activé la burbuja, pero seguía sintiéndome desprotegida, siempre pensé que este invento había sido la revolución del siglo, que finalmente me permitiría andar por las calles sin tapujos ni miedos. Pero el miedo tiene raíces de hierro, el miedo es un virus latente que se reproduce en el más enigmático páramo de la mente y se apodera de cada una de nuestras células dejándonos sin autonomía. Era consciente que ni llevando una armadura de acero podría pasearme por la periferia más peligrosa sintiéndome un ser invulnerable, todos los estigmas que me habían inculcado por mi condición de fémina seguían intactos, indelebles, eran las peores cicatrices con las que tendría que convivir.  Burbuja activada, música sintonizada y con el sobresalto en el pecho me dispuse a andar rápido, ya que la primera norma es la siguiente:  Nunca te quedes parada en un sitio desconocido, eso alienta a los más perversos.  Sí, lo sé, es triste pero es así. El horror recorría mi cuerpo, no dejaba de sentir unos ojos en mi nuca, me giré a cada instante comprobando que nadie me seguía, y de repente la música del dispositivo que iba conectado a mi oreja cambió y empezó a sonar : Every breath you take de The Police.  Os prometo que fue entonces, justo en ese instante, cuando empecé a dudar de mi cordura, muchos creen que esta canción es una bonita canción de amor, hay descerebrados que la han utilizado incluso en el baile de su boda, no es que yo sea una erudita en el amor, todo depende de la concepción que uno tenga, si para vosotros el amor es ser un acosador obsesivo, está es vuestra canción queridos. ``Every breath you take, every move you make, every step you take, I’ll be watching you. Oh can’t you see, you belong to me?'' ¿A quién no le alarmaría, si de repente, por arte de magia, apareciese esa canción en su reproductor coclear? Me volví a girar, y esta vez había un hombre de unos 50 años, sentado en un banco leyendo un libro, alzó la vista, me miró con perplejidad y volvió sumergirse en su relato. Mis pulmones se habían hecho repelentes al aire, me faltaba el aliento pero tenía que seguir para poder llegar a mi destino. Esa fue mi primera experiencia de scopaesthesia. La scopaesthesia, para quien no lo sepa, es un fenómeno que afirma que los seres humanos somos capaces de detectar si alguien nos esta mirando por la espalda. Es una sensación extrasensorial que aún hoy en día sigue siendo un misterio y que puede explicarse de la siguiente forma: cuando sientes que alguien te está mirando, al girarte, fuerzas a esa persona a devolverte la mirada, y ese hecho da la falsa sensación de que te están mirando o así me lo quisieron vender la infinidad de médicos a los que he visitado. Este suceso lleva repitiéndose diariamente desde aquel infame día en el que me desperté con ese ataque de pánico cuya explicación desconocía hasta hoy mismo. 