\begin{em}
``\ (\textellipsis) La predicción no es realmente ajustada, pero sí lo suficientemente certera como para tenerla en cuenta. No quiero conducir, con este documento, a equívocos. Parece que hablemos de algo sumamente imposible de deducir pero, con una pericia suficiente en estadística, un estudio personal histórico y un ordenador cuántico con buena capacidad\textellipsis\ cualquiera puede llegar a calcularlo. Arthur C. Clarke dio la gran sentencia de que cualquier tecnología lo suficientemente avanzada es indistinguible de la magia. Créame que no debe haber ni asomo de duda ni de oscurantismo y mucho menos de misticismo alrededor de este estudio. La tecnología avanzada parece magia para aquellos que no lo entienden, para el resto solo es un instrumento con cierta utilidad. Es evidente que podemos estar hablando de un cambio significativo a todos los niveles, sobre todo en el social, pero de nuevo: no tema el cambio, simplemente úselo y compréndalo. 

Debemos remontarnos a los análisis toscos de los peritos del seguro. ¿Sabe que cada persona tenía una numeración acorde a su riesgo? No es de extrañar. Una compañía de seguros solo acogerá a aquel que no suponga un riesgo monetario y pueda alimentar sus famélicos bolsillos durante todos los años posibles. Para esos análisis se requería un informe médico, con el historial de enfermedades de los antepasados que se hayan podido heredar, sumando el trabajo actual; el cual a su vez tiene su propia prima de riesgo… un sinfín de detalles que daban como resultado un número de peligro. Cuando más bajo fuera, más asegurable era uno. Es innegable que se fue perfeccionando con el tiempo, pero nunca al nivel que hemos alcanzado hoy en día: hoy en día podemos deducir el día de su muerte. 

Todos vamos a morir, al menos en nuestra forma biológica. Con el tiempo podremos advertir la hora exacta e incluso la causa, directa o indirecta, de su defunción. De ahí que me hayan pedido redactar este informe: necesitamos fondos para poder continuar con la investigación y usted es un perfil deseable. Incluimos en un sobre sellado, para que la decisión recaiga en su persona, el día de su fallecimiento\textellipsis\ considérelo una muestra de buena praxis por nuestra parte. (\textellipsis)''
\end{em}