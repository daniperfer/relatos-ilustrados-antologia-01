La ciudad, muda como estaba, componía una visión estremecedora, de una extrañeza sucinta, tan lejana del agrio tumulto. Recorrieron los pasajes estrechos que fluían paralelos a la avenida principal, a través de interiores destechados, al final de los cuales se abría un rectángulo de luz blanca, parcas nubes y un cielo amargo, evanescente. 

Era posible que las ruinas albergasen habitantes solitarios o pequeñas bandas y familias. Si se daba el caso, debía procederse a identificar a cada uno, reduciendo a los hostiles, para informar una vez terminada la misión. Se establecían puntos de recogida, pero no se entregaban recursos ni ayuda humanitaria a los que permanecieran en las ruinas. Los landers tenían prohibido entregar bolsas de agua, pastillas potabilizadoras o alguna de sus galletas. Pero no era habitual encontrar supervivientes entre los restos de las ciudades arrasadas. Lo fue por un tiempo, hasta que el bando rival desarrolló las unidades articuladas de rastreo: robots autónomos que podían correr y sortear obstáculos, ideales para introducirse en los bloques de edificios para hacer la limpieza. Si en algún momento se topaban con una unidad enemiga, las instrucciones eran claras: ocultarse y destruir. Se sabía que los robots rastreadores estaban provistos de cámaras de infrarrojos, sistemas de detección acústica además de un radar de partículas que funcionaba en base a los datos recogidos por un filtro de aire, por lo que estas unidades podían olfatear un rastro para perseguir a sus presas. Por eso los llamaban perros. Pol se preguntaba si su compañera habría tenido algún encuentro con uno. Por lo que había oído, Frida había tenido que abandonar Mercado, la ciudad donde vivía, junto a una turba de cientos de miles de refugiados. De su periplo hasta alistarse en el Cuerpo de Mensajeros no se sabía nada.