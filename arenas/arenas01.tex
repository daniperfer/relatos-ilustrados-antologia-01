Peleas a muerte ha habido siempre. Entre humanos, entre perros, gallos, coyotes, robots, naves. Me imagino que es una forma de entretenernos pensando que hay algo en juego de verdad. Y siempre ha habido apuestas en ellas. Algunos veíamos en ello una forma apasionante de jugarnos nuestro dinero. 

Pero las arenas, lo que ocurre en las arenas es algo totalmente diferente. No sé a quién se le ocurrió el sistema, pero es lo más adictivo a lo que me he enfrentado jamás y lo digo habiendo sido adicto a un montón de cosas. No se apuesta al vencedor del combate. Se apuesta si el que muere es humano o Sint. Imagínate. ¿Cómo saberlo? He visto morir a mujeres con tres brazos que eran humanas y Sints desangrados por cortes de espada como si fuesen gladiadores romanos. 

Es impresionante. En cuanto lo descubrí me volví loco. Hasta llegué a participar como luchador. Le pedí a papá que me pagase un buen maestro de artes marciales. Conseguimos uno de los mejores: Jim Sheridan, el encargado de la seguridad de los mandatarios de la RíoCorp. Cada vez que recuerdo sus clases me estremezco. 

Ni siquiera llegué a verle en persona, pero de haberlo hecho no sé cómo habría reaccionado. Jamás he sentido tanta admiración por nadie. Durante 54 semanas estuvimos compartiendo espacio virtual. Es un tipo súper detallista, en función de cada lección recreaba un ambiente virtual que fuese acorde. Cuando hablábamos del manejo del kendo nos metíamos en un domo japonés y nos vestía con los ropajes clásicos. Cuando hablábamos de artes marciales europeas y del manejo de distintas espadas, nos llevaba hasta un castillo europeo de mitad del siglo XV. Creo que nunca había disfrutado tanto. Recuerdo vagamente las lecciones de español a las que me apuntó papá cuando tenía ocho o nueve años, pero creo que me enganché a ellas porque la profesora me ponía tanto que daba igual lo que me estuviese enseñando, le habría hecho caso de cualquier modo. 

Tras más de un año de entrenamiento creí que estaba preparado para competir en las arenas. Pero no me metí en la liga oficial, nadie se mete en una competición de este tipo para acabar en la versión bonita y televisada. Si haces algo así es mejor que empieces por lo clandestino, que no sólo puedas morir durante el combate, que puedas hacerlo simplemente por estar allí, que hasta pegarte mucho a las paredes pueda mandarte al otro barrio.

Fue mi amigo Horacio el que me consiguió un combate en un viejo teatro abandonado. Él siempre me había acompañado desde el colegio. Robamos chucherías juntos, estuvimos juntos en el equipo de NekBall del instituto y participamos juntos en el programa formativo de CorpBleund en Holanda. Siempre estuvimos tan unidos que mi padre le aceptó como si fuese un miembro más del Clan, a pesar de no ser un Kaeda y de su aspecto de paliducho occidental.

Algunos hablan sin parar de sus buenos amigos, aquellos que siempre están ahí para dar un buen consejo y actúan como si fuesen una parte de tu conciencia. Horacio no es así. Horacio siempre me ha azuzado y me ha lanzado contra toda clase de peligros. Se lo agradezco, porque de no haber estado a mi lado mi vida hubiese sido mucho más aburrida. Cuando le conté que me estaba entrenando para participar en las arenas, se quitó las gafas, me miró en silencio durante diez minutos y grito: ``sí, joder, sí, vamos a ganar un montón de pasta''.