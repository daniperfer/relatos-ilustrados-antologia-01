Un estallido retumbó en el cielo, que vibró a punto de romperse. El traje de latexkín de Frida reflejó por un momento el relámpago que se produjo cuando el androide de cinco metros de altura se alejó, entrando en match 1. El encuentro con Deucalión había sido breve; como en todas las misiones de campo, se había limitado a unos minutos, lo suficiente para el intercambio de datos o recibir instrucciones nuevas para continuar. La nube de polvo, resultado del despegue vertical, envolvía a Pol y Frida, dos siluetas borrosas, pertrechados con ceñidos trajes de supervivencia de latexkín negro y verde con las insignias del Cuerpo de Mensajeros. 

Pol acabó de escribir las notas en su bitácora, con una rodilla apoyada en el suelo, y guardó el pequeño cuaderno y la pluma en uno de los bolsillos laterales del traje. Unos metros atrás, Frida se balanceaba sobre sus fuertes piernas, mirando al frente y sosteniendo el fusil de impulsos, con las pupilas fijas bajo el flequillo recto. Desde que se puso en marcha el programa, las salidas se organizaban en parejas, con un agente especializado en comunicaciones y un soldado encargado de la seguridad y la orientación, ambos armados. 
\reply ¿Tienes el rollo?

Asintió de espaldas a Frida, observando el estuche que había extraído del muslo acorazado de Deucalión. Despegó suavemente las dos tiras de velcro de los extremos y extendió el hatillo. El forro interior de material térmico albergaba un tallo de fibra óptica recubierto de aislante. El thalus combinaba el tejido orgánico con cristales y fibra óptica, capaces de crecer en ramificaciones a medida que acumulaban información: un árbol de datos. 

Pol encendió el visor y se ciñó la cinta ajustable alrededor de la cabeza. Esperó el OK del display en el cristal RV del dispositivo y en cuanto lo tuvo, apagó el visor y se lo colocó sobre la frente.
\reply Validación aceptada.
 
Frida se colgó el fusil al hombro, esperó a que Pol cerrara de nuevo el estuche y le ayudó a guardárselo en la mochila adherida al traje.