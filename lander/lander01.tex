Un estallido retumbó en el cielo, que vibró a punto de romperse. El traje de latexkín de Frida reflejó por un momento el relámpago que se produjo cuando el androide de cinco metros de altura se alejó, entrando en match 1. El encuentro con Deucalión había sido breve; como en todas las misiones de campo, se había limitado a unos minutos, lo suficiente para el intercambio de datos o recibir instrucciones nuevas para continuar. La nube de polvo, resultado del despegue vertical, envolvía a Pol y Frida, dos siluetas borrosas, pertrechados con ceñidos trajes de supervivencia de latexkín negro y verde con las insignias del Cuerpo de Mensajeros. 

Pol acabó de escribir las notas en su bitácora, con una rodilla apoyada en el suelo, y guardó el pequeño cuaderno y la pluma en uno de los bolsillos laterales del traje. Unos metros atrás, Frida se balanceaba sobre sus fuertes piernas, mirando al frente y sosteniendo el fusil de impulsos, con las pupilas fijas bajo el flequillo recto. Desde que se puso en marcha el programa, las salidas se organizaban en parejas, con un agente especializado en comunicaciones y un soldado encargado de la seguridad y la orientación, ambos armados. 
\reply ¿Tienes el rollo?

Asintió de espaldas a Frida, observando el estuche que había extraído del muslo acorazado de Deucalión. Despegó suavemente las dos tiras de velcro de los extremos y extendió el hatillo. El forro interior de material térmico albergaba un tallo de fibra óptica recubierto de aislante. El thalus combinaba el tejido orgánico con cristales y fibra óptica, capaces de crecer en ramificaciones a medida que acumulaban información: un árbol de datos. 

Pol encendió el visor y se ciñó la cinta ajustable alrededor de la cabeza. Esperó el OK del display en el cristal RV del dispositivo y en cuanto lo tuvo, apagó el visor y se lo colocó sobre la frente.
\reply Validación aceptada.
 
Frida se colgó el fusil al hombro, esperó a que Pol cerrara de nuevo el estuche y le ayudó a guardárselo en la mochila adherida al traje.

Abandonaron el claro y se adentraron en un pinar, cruzando por un sendero parcheado de vegetación baja y manchado por los huecos de luz que se colaban entre las ramas. Las botas de Frida hacían crujir la maleza a un ritmo constante y Pol las seguía hipnotizado, sintiendo cómo el latexkín dispersaba el calor acumulado por todo el traje. A pesar de la misión, Pol disfrutaba del contacto con la naturaleza, entrando en un estado de dulce sopor. Se preguntaba si Frida sentiría lo mismo, si ella también agradecía sentir el sol en la piel y el aire fresco después de tanto tiempo viviendo en la penumbra y la saturación de los búnkeres. Después de todo, habían sido de las primeras patrullas en acometer las misiones en campo abierto, desde que el programa de mensajeros se pusiera en marcha, hacía diez meses. 

El terreno cambió bruscamente, mostrando una suave inclinación que formaba una pequeña ladera. Realizaron el descenso y se detuvieron en la linde, contemplando el principio de las ruinas: los edificios derruidos se apiñaban en una amarga sonrisa desdentada. Las capitales que no habían sido devastadas por la artillería pesada o los láseres impulsados por alta energía, se habían convertido durante la guerra en escenario de escaramuzas y toma de posiciones de los ejércitos. La principal misión de los landers del Cuerpo de Mensajeros era cruzar las ciudades a pie hasta un punto determinado para entregar el árbol de datos y regresar al punto de control. Una vez de vez de vuelta, esperarían a ser recogidos por un vehículo. 

Adoptando la postura de seguridad, agachada, con el fusil enfilado hacia la ciudad, Frida realizó un barrido visual para cerciorarse de que el terreno estaba despejado. Al cabo de unos minutos se levantó y retomó el paso. Para Pol, esa forma de andar no era tanto la de una soldado, como la de la propia Frida, la persona más decidida y resolutiva que conocía.

Continuaron la marcha en silencio. El entrenamiento al que eran sometidos los agentes era muy estricto en eso. Pol ya sabía que aquel mutismo se alargaría durante horas y que solo les acompañaría el sonido de las pisadas y los cristales rotos, tan diferente al paso mullido de la tierra húmeda del bosque. Una amplia nube ocultó el sol y otra vez sintió la regulación de temperatura por todo su cuerpo. La vista de la ciudad se hacía más grande y cobraba tridimensionalidad a medida que avanzaban, pasando de ser una lámina aguada y difusa a lo que parecía una minuciosa reconstrucción a escala. Llegaron al comienzo de una avenida donde las líneas de la carretera se habían borrado. El asfalto se mostraba salpicado de calvas donde el gramón crecía seco y amarillento. Sortearon una hilera de coches. Las berlinas de gruesa carrocería tenían los neumáticos hundidos por el peso. Sobre los parabrisas, los que no estaban rotos o agrietados, una capa de polvo brillante se apelmazaba en pequeñas escamas. Allí donde el lacado de la superficie había saltado, el aluminio asomaba sucio y desgastado, y el acero se cubría de líquenes abonados por heces de aves y roedores. Pol identificó los últimos modelos construidos hacía una década, testimonio del fin de toda producción industrial que no estuviera dedicada a la guerra. 

La puerta de uno de los vehículos tembló y se abatió hasta abrirse completamente. Al instante Frida se dio la vuelta: Pol vio sus ojos fijos bajo el flequillo tratando de identificar el objetivo, antes de que el fogonazo del fusil le deslumbrase completamente. La oyó gritar “abajo, abajo” y luego sintió la lluvia de cristales caer sobre su espalda y rebotar en el latexkín endurecido, que había formado una carcasa rígida. El tufo a pelo y carne quemada se esparció por el aire al mismo tiempo que un crepitar químico se extinguía. Cuando Pol alzó la vista, pudo ver cómo Frida se acercaba en posición de combate al vehículo y propinaba una patada a la puerta calcinada, que cayó al suelo. La cabeza de un ciervo joven, con el astado sin desarrollar colgaba del asiento trasero, casi separada del cuerpo, renegrida y con los ojos cocidos al instante por el calor del impacto. El forro del techo también estaba quemado y el cuero sintético del asiento se había fundido con la piel del animal. Todos los vidrios habían saltado por los aires, creando una alfombra de cristal y sedimentos de sangre seca. Los fusiles de impulso eran armas silenciosas, ideales para resultar indetectables en las misiones de campo y su potencia era regulable. Estaba claro que Frida no esperaba tener que disparar por el momento, menos crear un estruendo por cruzarse con un ciervo en el camino, y eso la iba a poner de mal humor, por lo que Pol sabía que más adelante iba a necesitar una descarga. Pero no lo iba a exteriorizar. Frida no.

\reply Hay que despiezarlo. Rápido. Nos servirá para la cena.

\reply ¿Crees que podrían estar cerca...?

\reply ¡Shhh! \pause le interrumpió Frida\pauseend. ¡Calla! Empieza a cortarlo. Rápido.

Ningún lander podía decir que le gustaran las galletas y barritas que el Cuerpo de Mensajeros facilitaba para las misiones, así que no era raro que cazaran algún animalillo y se aprovechara un alto en el camino para degustar un bocado de  carne. La presa quedaba asada automáticamente por la ráfaga del láser. Pol se apresuró a cortar las extremidades de la pieza con un pequeño vibrofilo. Las metió en bolsas de cierre hermético, enrollándolas para atarlas todas juntas y colgárselas a la espalda. 

No tardaron en reanudar la marcha. Ante ellos, la calle principal se prolongaba en una decadente repetición de semáforos inertes y ventanas vacías. Al final de la avenida, las colosales arcologías se alzaban soñolientas, coronadas por el verde apagado de los jardines en los niveles residenciales. Una maraña de enredaderas se abrazaba a los balcones de cemento y pendía de las plataformas de los párquines elevados. Los soportales plásticos goteaban aquí y allá, filtrando la humedad condensada antes del amanecer. Bajo estos, la luz difusa dibujaba las vaporosas siluetas de los escombros. 

\reply Sigamos por aquí \pause sentenció Frida, que giró hacia calles secundarias, a sabiendas de que por la avenida principal podrían ser observados fácilmente.